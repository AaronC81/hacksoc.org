\documentclass[12pt,a4paper]{article}
\pagestyle{headings}

\usepackage[utf8x]{inputenc}
\usepackage{ucs}
\usepackage[british]{babel}
\usepackage[nodayofweek]{datetime}
\usepackage{hyperref}

\author{Michael Walker}
\title{Version Control and Teamwork}

\begin{document}

\maketitle
\tableofcontents
\pagebreak

\section{Version Control}

A version control system (VCS) is a tool (or collection of tools)
which make working on multiple versions of something, be it a program,
website, document, whatever, and keeping track of all of the changes
much easier. If it helps, you can think of a VCS as a glorified undo
tool.

If you're used to developing software by first making a backup of the
file you want to work on, and then editing the original file (or even
just editing the original file); or by editing the live system rather
than an offline copy; you need version control.

The work-flow for version control is a bit different:

\begin{description}
  \item[Edit] the appropriate files, there is no need to take a backup
    first.
  \item[Test] your changes. Of course, you did this before, right?
  \item[Commit] your changes, that is, tell the VCS which files you
    have changed and (usually) enter a short message explaining what
    you did.
  \item[Deploy] your commits, that is, send your commits to a server,
    or some other user. If you're the sole user of the VCS for your
    project, this may not be necessary (but can still be good---more
    on that later).
\end{description}

You may think that having to enter a message for each commit is a
hassle. Well, consider that by entering a good message, you'll be able
to see what the commit did. This may be incredibly useful down the
line when you're trying to find out where you introduced a bug, or
removed something that was needed. By using a VCS, you're not just
giving yourself an easy way to undo things, you're building a
detailed history of the development of your project, which is an
invaluable tool in itself.

Eventually, you will make a commit that you realise was a bad
idea. It's inevitable. It's ok. You can tell people you were drunk
when you did it. Fortunately, we come to another strength of VCSs
here, undoing things. Every VCS (that I know of) has some way to tell
the system to undo a specific commit. How this is implemented
varies---the commit may simply vanish, or the system may insert a new
commit reverting all of its changes. You may also need to do some
manual intervention, if you're undoing something which you have since
built upon.

Finally, let's cover branches and merging in this introduction. These
are key concepts in the VCS workflow, and entire development paradigms
have sprung up around how to use branches correctly (we'll be looking
at two). A branch is a copy of your code. Each branch can be distinct,
and branches can be merged (perhaps with some manual intervention
required) at any time as well. It's possibly easiest to explain with
a couple of examples of typical branches you may have:

\begin{description}
  \item[master] this is typically your stable, production-ready
    code.
  \item[devel] depending on your workflow, you may also have a
    development branch. This will be ahead of master, and will be
    merged into master when you have implemented everything you want
    for the next release
  \item[hotfixes] again, depending on your workflow, you may have a
    branch for emergency fixes to releases. This will also be ahead of
    master, but only by a handful of commits at a time, after which
    point it will be merged and a new release made.
\end{description}

Of course, where branches really come into their own is when you have
a good workflow based around them. It's also typical to have ``feature
branches'', one branch for each new feature you are implementing,
which is them merged into the development branch when complete.

\subsection{Distributed vs. Centralised}

There are two approaches when it comes to designing version control
systems, and which one your system uses will drastically affect how
you work with it. These approaches are distributed (or decentralised)
and centralised systems.

\begin{description}
  \item[Centralised] systems have one central server which houses the
    repository. If you commit something, it is sent to the server, if
    you lose access for a period of time you can't keep working and
    deploy your commits later. Despite this downside, because everyone
    is working off the same copy of the code, conflicts when merging
    may be reduced.
  \item[Distributed] systems do not have a central server. If you
    commit something, it is saved on your local machine. When you
    deploy your commits, that simply consists of you sending them to
    another user, who merges them into their copy. As there is no
    server to lose access to, you can keep working at all
    times. However, because everyone has their own version of the
    code, which may have many commits difference with someone else,
    merge conflicts may be more common.
\end{description}

In practice, distributed systems are much more convenient to use than
centralised systems, and the potential problem of very inconsistent
repositories is not often realised. Furthermore, users typically use a
hybrid of the two approaches, a distributed system with one
``canonical'' version of the code, which may be guarded by one
privileged user. There may even be multiple levels of this canonical
code, forming a tree structure with trusted users at each level
accepting things from the level below which pass the quality checks,
possibly doing some more work of their own, and then pushing their
code up the tree to be reviewed by the next person. This sort of
structure is often used in large open source projects, for example,
the Linux kernel. Linus Torvalds has control over the canonical kernel
code, and reviews all commits to it and either accepts or denies.

A typical workflow for distributed systems is to have an account on a
service like GitHub, which provides a globally-accessible repository
which you can push your code to. This allows you to work anywhere even
if you didn't bring your local copy of the code with you, provides
free backup of all of your commits, and enables people working with
you to frequently merge your commits and so reduce potential
conflicts.

\subsection{Some popular VCS tools}

There are a lot of version control tools. Below are some of the major
ones, which is it often convenient to have at least a basic knowledge
of, even if you use one of them exclusively for everything you do:

\begin{description}
  \item[Concurrent Versions System (CVS)] centralised, very old.
  \item[Subversion (svn)] centralised, intended to be a better CVS.
  \item[Git] distributed, has gained a lot of popularity in the open
    source community.
  \item[Mercurial (hg)] distributed, has gained some popularity in the
    open source community.
\end{description}

I find that centralised tools are usually used in organisations with a
strict hierarchy, and distributed tools in organisations where every
developer is (more or less) equal. If nothing else, an interesting
modern manifestation of how the Cathedral and the Bazaar methods
differ.

If you're going to join us in working on the DDNS project, we shall be
using Git and GitHub extensively.

\section{Using Git}

\subsection{Git Flow}

\subsection{Github Flow}

\section{Some Best Practices for Teamwork}

\end{document}